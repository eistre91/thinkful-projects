
% Default to the notebook output style

    


% Inherit from the specified cell style.




    
\documentclass[11pt]{article}

    
    
    \usepackage[T1]{fontenc}
    % Nicer default font (+ math font) than Computer Modern for most use cases
    \usepackage{mathpazo}

    % Basic figure setup, for now with no caption control since it's done
    % automatically by Pandoc (which extracts ![](path) syntax from Markdown).
    \usepackage{graphicx}
    % We will generate all images so they have a width \maxwidth. This means
    % that they will get their normal width if they fit onto the page, but
    % are scaled down if they would overflow the margins.
    \makeatletter
    \def\maxwidth{\ifdim\Gin@nat@width>\linewidth\linewidth
    \else\Gin@nat@width\fi}
    \makeatother
    \let\Oldincludegraphics\includegraphics
    % Set max figure width to be 80% of text width, for now hardcoded.
    \renewcommand{\includegraphics}[1]{\Oldincludegraphics[width=.8\maxwidth]{#1}}
    % Ensure that by default, figures have no caption (until we provide a
    % proper Figure object with a Caption API and a way to capture that
    % in the conversion process - todo).
    \usepackage{caption}
    \DeclareCaptionLabelFormat{nolabel}{}
    \captionsetup{labelformat=nolabel}

    \usepackage{adjustbox} % Used to constrain images to a maximum size 
    \usepackage{xcolor} % Allow colors to be defined
    \usepackage{enumerate} % Needed for markdown enumerations to work
    \usepackage{geometry} % Used to adjust the document margins
    \usepackage{amsmath} % Equations
    \usepackage{amssymb} % Equations
    \usepackage{textcomp} % defines textquotesingle
    % Hack from http://tex.stackexchange.com/a/47451/13684:
    \AtBeginDocument{%
        \def\PYZsq{\textquotesingle}% Upright quotes in Pygmentized code
    }
    \usepackage{upquote} % Upright quotes for verbatim code
    \usepackage{eurosym} % defines \euro
    \usepackage[mathletters]{ucs} % Extended unicode (utf-8) support
    \usepackage[utf8x]{inputenc} % Allow utf-8 characters in the tex document
    \usepackage{fancyvrb} % verbatim replacement that allows latex
    \usepackage{grffile} % extends the file name processing of package graphics 
                         % to support a larger range 
    % The hyperref package gives us a pdf with properly built
    % internal navigation ('pdf bookmarks' for the table of contents,
    % internal cross-reference links, web links for URLs, etc.)
    \usepackage{hyperref}
    \usepackage{longtable} % longtable support required by pandoc >1.10
    \usepackage{booktabs}  % table support for pandoc > 1.12.2
    \usepackage[inline]{enumitem} % IRkernel/repr support (it uses the enumerate* environment)
    \usepackage[normalem]{ulem} % ulem is needed to support strikethroughs (\sout)
                                % normalem makes italics be italics, not underlines
    

    
    
    % Colors for the hyperref package
    \definecolor{urlcolor}{rgb}{0,.145,.698}
    \definecolor{linkcolor}{rgb}{.71,0.21,0.01}
    \definecolor{citecolor}{rgb}{.12,.54,.11}

    % ANSI colors
    \definecolor{ansi-black}{HTML}{3E424D}
    \definecolor{ansi-black-intense}{HTML}{282C36}
    \definecolor{ansi-red}{HTML}{E75C58}
    \definecolor{ansi-red-intense}{HTML}{B22B31}
    \definecolor{ansi-green}{HTML}{00A250}
    \definecolor{ansi-green-intense}{HTML}{007427}
    \definecolor{ansi-yellow}{HTML}{DDB62B}
    \definecolor{ansi-yellow-intense}{HTML}{B27D12}
    \definecolor{ansi-blue}{HTML}{208FFB}
    \definecolor{ansi-blue-intense}{HTML}{0065CA}
    \definecolor{ansi-magenta}{HTML}{D160C4}
    \definecolor{ansi-magenta-intense}{HTML}{A03196}
    \definecolor{ansi-cyan}{HTML}{60C6C8}
    \definecolor{ansi-cyan-intense}{HTML}{258F8F}
    \definecolor{ansi-white}{HTML}{C5C1B4}
    \definecolor{ansi-white-intense}{HTML}{A1A6B2}

    % commands and environments needed by pandoc snippets
    % extracted from the output of `pandoc -s`
    \providecommand{\tightlist}{%
      \setlength{\itemsep}{0pt}\setlength{\parskip}{0pt}}
    \DefineVerbatimEnvironment{Highlighting}{Verbatim}{commandchars=\\\{\}}
    % Add ',fontsize=\small' for more characters per line
    \newenvironment{Shaded}{}{}
    \newcommand{\KeywordTok}[1]{\textcolor[rgb]{0.00,0.44,0.13}{\textbf{{#1}}}}
    \newcommand{\DataTypeTok}[1]{\textcolor[rgb]{0.56,0.13,0.00}{{#1}}}
    \newcommand{\DecValTok}[1]{\textcolor[rgb]{0.25,0.63,0.44}{{#1}}}
    \newcommand{\BaseNTok}[1]{\textcolor[rgb]{0.25,0.63,0.44}{{#1}}}
    \newcommand{\FloatTok}[1]{\textcolor[rgb]{0.25,0.63,0.44}{{#1}}}
    \newcommand{\CharTok}[1]{\textcolor[rgb]{0.25,0.44,0.63}{{#1}}}
    \newcommand{\StringTok}[1]{\textcolor[rgb]{0.25,0.44,0.63}{{#1}}}
    \newcommand{\CommentTok}[1]{\textcolor[rgb]{0.38,0.63,0.69}{\textit{{#1}}}}
    \newcommand{\OtherTok}[1]{\textcolor[rgb]{0.00,0.44,0.13}{{#1}}}
    \newcommand{\AlertTok}[1]{\textcolor[rgb]{1.00,0.00,0.00}{\textbf{{#1}}}}
    \newcommand{\FunctionTok}[1]{\textcolor[rgb]{0.02,0.16,0.49}{{#1}}}
    \newcommand{\RegionMarkerTok}[1]{{#1}}
    \newcommand{\ErrorTok}[1]{\textcolor[rgb]{1.00,0.00,0.00}{\textbf{{#1}}}}
    \newcommand{\NormalTok}[1]{{#1}}
    
    % Additional commands for more recent versions of Pandoc
    \newcommand{\ConstantTok}[1]{\textcolor[rgb]{0.53,0.00,0.00}{{#1}}}
    \newcommand{\SpecialCharTok}[1]{\textcolor[rgb]{0.25,0.44,0.63}{{#1}}}
    \newcommand{\VerbatimStringTok}[1]{\textcolor[rgb]{0.25,0.44,0.63}{{#1}}}
    \newcommand{\SpecialStringTok}[1]{\textcolor[rgb]{0.73,0.40,0.53}{{#1}}}
    \newcommand{\ImportTok}[1]{{#1}}
    \newcommand{\DocumentationTok}[1]{\textcolor[rgb]{0.73,0.13,0.13}{\textit{{#1}}}}
    \newcommand{\AnnotationTok}[1]{\textcolor[rgb]{0.38,0.63,0.69}{\textbf{\textit{{#1}}}}}
    \newcommand{\CommentVarTok}[1]{\textcolor[rgb]{0.38,0.63,0.69}{\textbf{\textit{{#1}}}}}
    \newcommand{\VariableTok}[1]{\textcolor[rgb]{0.10,0.09,0.49}{{#1}}}
    \newcommand{\ControlFlowTok}[1]{\textcolor[rgb]{0.00,0.44,0.13}{\textbf{{#1}}}}
    \newcommand{\OperatorTok}[1]{\textcolor[rgb]{0.40,0.40,0.40}{{#1}}}
    \newcommand{\BuiltInTok}[1]{{#1}}
    \newcommand{\ExtensionTok}[1]{{#1}}
    \newcommand{\PreprocessorTok}[1]{\textcolor[rgb]{0.74,0.48,0.00}{{#1}}}
    \newcommand{\AttributeTok}[1]{\textcolor[rgb]{0.49,0.56,0.16}{{#1}}}
    \newcommand{\InformationTok}[1]{\textcolor[rgb]{0.38,0.63,0.69}{\textbf{\textit{{#1}}}}}
    \newcommand{\WarningTok}[1]{\textcolor[rgb]{0.38,0.63,0.69}{\textbf{\textit{{#1}}}}}
    
    
    % Define a nice break command that doesn't care if a line doesn't already
    % exist.
    \def\br{\hspace*{\fill} \\* }
    % Math Jax compatability definitions
    \def\gt{>}
    \def\lt{<}
    % Document parameters
    \title{GDELT Capstone Report}
    
    
    

    % Pygments definitions
    
\makeatletter
\def\PY@reset{\let\PY@it=\relax \let\PY@bf=\relax%
    \let\PY@ul=\relax \let\PY@tc=\relax%
    \let\PY@bc=\relax \let\PY@ff=\relax}
\def\PY@tok#1{\csname PY@tok@#1\endcsname}
\def\PY@toks#1+{\ifx\relax#1\empty\else%
    \PY@tok{#1}\expandafter\PY@toks\fi}
\def\PY@do#1{\PY@bc{\PY@tc{\PY@ul{%
    \PY@it{\PY@bf{\PY@ff{#1}}}}}}}
\def\PY#1#2{\PY@reset\PY@toks#1+\relax+\PY@do{#2}}

\expandafter\def\csname PY@tok@w\endcsname{\def\PY@tc##1{\textcolor[rgb]{0.73,0.73,0.73}{##1}}}
\expandafter\def\csname PY@tok@c\endcsname{\let\PY@it=\textit\def\PY@tc##1{\textcolor[rgb]{0.25,0.50,0.50}{##1}}}
\expandafter\def\csname PY@tok@cp\endcsname{\def\PY@tc##1{\textcolor[rgb]{0.74,0.48,0.00}{##1}}}
\expandafter\def\csname PY@tok@k\endcsname{\let\PY@bf=\textbf\def\PY@tc##1{\textcolor[rgb]{0.00,0.50,0.00}{##1}}}
\expandafter\def\csname PY@tok@kp\endcsname{\def\PY@tc##1{\textcolor[rgb]{0.00,0.50,0.00}{##1}}}
\expandafter\def\csname PY@tok@kt\endcsname{\def\PY@tc##1{\textcolor[rgb]{0.69,0.00,0.25}{##1}}}
\expandafter\def\csname PY@tok@o\endcsname{\def\PY@tc##1{\textcolor[rgb]{0.40,0.40,0.40}{##1}}}
\expandafter\def\csname PY@tok@ow\endcsname{\let\PY@bf=\textbf\def\PY@tc##1{\textcolor[rgb]{0.67,0.13,1.00}{##1}}}
\expandafter\def\csname PY@tok@nb\endcsname{\def\PY@tc##1{\textcolor[rgb]{0.00,0.50,0.00}{##1}}}
\expandafter\def\csname PY@tok@nf\endcsname{\def\PY@tc##1{\textcolor[rgb]{0.00,0.00,1.00}{##1}}}
\expandafter\def\csname PY@tok@nc\endcsname{\let\PY@bf=\textbf\def\PY@tc##1{\textcolor[rgb]{0.00,0.00,1.00}{##1}}}
\expandafter\def\csname PY@tok@nn\endcsname{\let\PY@bf=\textbf\def\PY@tc##1{\textcolor[rgb]{0.00,0.00,1.00}{##1}}}
\expandafter\def\csname PY@tok@ne\endcsname{\let\PY@bf=\textbf\def\PY@tc##1{\textcolor[rgb]{0.82,0.25,0.23}{##1}}}
\expandafter\def\csname PY@tok@nv\endcsname{\def\PY@tc##1{\textcolor[rgb]{0.10,0.09,0.49}{##1}}}
\expandafter\def\csname PY@tok@no\endcsname{\def\PY@tc##1{\textcolor[rgb]{0.53,0.00,0.00}{##1}}}
\expandafter\def\csname PY@tok@nl\endcsname{\def\PY@tc##1{\textcolor[rgb]{0.63,0.63,0.00}{##1}}}
\expandafter\def\csname PY@tok@ni\endcsname{\let\PY@bf=\textbf\def\PY@tc##1{\textcolor[rgb]{0.60,0.60,0.60}{##1}}}
\expandafter\def\csname PY@tok@na\endcsname{\def\PY@tc##1{\textcolor[rgb]{0.49,0.56,0.16}{##1}}}
\expandafter\def\csname PY@tok@nt\endcsname{\let\PY@bf=\textbf\def\PY@tc##1{\textcolor[rgb]{0.00,0.50,0.00}{##1}}}
\expandafter\def\csname PY@tok@nd\endcsname{\def\PY@tc##1{\textcolor[rgb]{0.67,0.13,1.00}{##1}}}
\expandafter\def\csname PY@tok@s\endcsname{\def\PY@tc##1{\textcolor[rgb]{0.73,0.13,0.13}{##1}}}
\expandafter\def\csname PY@tok@sd\endcsname{\let\PY@it=\textit\def\PY@tc##1{\textcolor[rgb]{0.73,0.13,0.13}{##1}}}
\expandafter\def\csname PY@tok@si\endcsname{\let\PY@bf=\textbf\def\PY@tc##1{\textcolor[rgb]{0.73,0.40,0.53}{##1}}}
\expandafter\def\csname PY@tok@se\endcsname{\let\PY@bf=\textbf\def\PY@tc##1{\textcolor[rgb]{0.73,0.40,0.13}{##1}}}
\expandafter\def\csname PY@tok@sr\endcsname{\def\PY@tc##1{\textcolor[rgb]{0.73,0.40,0.53}{##1}}}
\expandafter\def\csname PY@tok@ss\endcsname{\def\PY@tc##1{\textcolor[rgb]{0.10,0.09,0.49}{##1}}}
\expandafter\def\csname PY@tok@sx\endcsname{\def\PY@tc##1{\textcolor[rgb]{0.00,0.50,0.00}{##1}}}
\expandafter\def\csname PY@tok@m\endcsname{\def\PY@tc##1{\textcolor[rgb]{0.40,0.40,0.40}{##1}}}
\expandafter\def\csname PY@tok@gh\endcsname{\let\PY@bf=\textbf\def\PY@tc##1{\textcolor[rgb]{0.00,0.00,0.50}{##1}}}
\expandafter\def\csname PY@tok@gu\endcsname{\let\PY@bf=\textbf\def\PY@tc##1{\textcolor[rgb]{0.50,0.00,0.50}{##1}}}
\expandafter\def\csname PY@tok@gd\endcsname{\def\PY@tc##1{\textcolor[rgb]{0.63,0.00,0.00}{##1}}}
\expandafter\def\csname PY@tok@gi\endcsname{\def\PY@tc##1{\textcolor[rgb]{0.00,0.63,0.00}{##1}}}
\expandafter\def\csname PY@tok@gr\endcsname{\def\PY@tc##1{\textcolor[rgb]{1.00,0.00,0.00}{##1}}}
\expandafter\def\csname PY@tok@ge\endcsname{\let\PY@it=\textit}
\expandafter\def\csname PY@tok@gs\endcsname{\let\PY@bf=\textbf}
\expandafter\def\csname PY@tok@gp\endcsname{\let\PY@bf=\textbf\def\PY@tc##1{\textcolor[rgb]{0.00,0.00,0.50}{##1}}}
\expandafter\def\csname PY@tok@go\endcsname{\def\PY@tc##1{\textcolor[rgb]{0.53,0.53,0.53}{##1}}}
\expandafter\def\csname PY@tok@gt\endcsname{\def\PY@tc##1{\textcolor[rgb]{0.00,0.27,0.87}{##1}}}
\expandafter\def\csname PY@tok@err\endcsname{\def\PY@bc##1{\setlength{\fboxsep}{0pt}\fcolorbox[rgb]{1.00,0.00,0.00}{1,1,1}{\strut ##1}}}
\expandafter\def\csname PY@tok@kc\endcsname{\let\PY@bf=\textbf\def\PY@tc##1{\textcolor[rgb]{0.00,0.50,0.00}{##1}}}
\expandafter\def\csname PY@tok@kd\endcsname{\let\PY@bf=\textbf\def\PY@tc##1{\textcolor[rgb]{0.00,0.50,0.00}{##1}}}
\expandafter\def\csname PY@tok@kn\endcsname{\let\PY@bf=\textbf\def\PY@tc##1{\textcolor[rgb]{0.00,0.50,0.00}{##1}}}
\expandafter\def\csname PY@tok@kr\endcsname{\let\PY@bf=\textbf\def\PY@tc##1{\textcolor[rgb]{0.00,0.50,0.00}{##1}}}
\expandafter\def\csname PY@tok@bp\endcsname{\def\PY@tc##1{\textcolor[rgb]{0.00,0.50,0.00}{##1}}}
\expandafter\def\csname PY@tok@fm\endcsname{\def\PY@tc##1{\textcolor[rgb]{0.00,0.00,1.00}{##1}}}
\expandafter\def\csname PY@tok@vc\endcsname{\def\PY@tc##1{\textcolor[rgb]{0.10,0.09,0.49}{##1}}}
\expandafter\def\csname PY@tok@vg\endcsname{\def\PY@tc##1{\textcolor[rgb]{0.10,0.09,0.49}{##1}}}
\expandafter\def\csname PY@tok@vi\endcsname{\def\PY@tc##1{\textcolor[rgb]{0.10,0.09,0.49}{##1}}}
\expandafter\def\csname PY@tok@vm\endcsname{\def\PY@tc##1{\textcolor[rgb]{0.10,0.09,0.49}{##1}}}
\expandafter\def\csname PY@tok@sa\endcsname{\def\PY@tc##1{\textcolor[rgb]{0.73,0.13,0.13}{##1}}}
\expandafter\def\csname PY@tok@sb\endcsname{\def\PY@tc##1{\textcolor[rgb]{0.73,0.13,0.13}{##1}}}
\expandafter\def\csname PY@tok@sc\endcsname{\def\PY@tc##1{\textcolor[rgb]{0.73,0.13,0.13}{##1}}}
\expandafter\def\csname PY@tok@dl\endcsname{\def\PY@tc##1{\textcolor[rgb]{0.73,0.13,0.13}{##1}}}
\expandafter\def\csname PY@tok@s2\endcsname{\def\PY@tc##1{\textcolor[rgb]{0.73,0.13,0.13}{##1}}}
\expandafter\def\csname PY@tok@sh\endcsname{\def\PY@tc##1{\textcolor[rgb]{0.73,0.13,0.13}{##1}}}
\expandafter\def\csname PY@tok@s1\endcsname{\def\PY@tc##1{\textcolor[rgb]{0.73,0.13,0.13}{##1}}}
\expandafter\def\csname PY@tok@mb\endcsname{\def\PY@tc##1{\textcolor[rgb]{0.40,0.40,0.40}{##1}}}
\expandafter\def\csname PY@tok@mf\endcsname{\def\PY@tc##1{\textcolor[rgb]{0.40,0.40,0.40}{##1}}}
\expandafter\def\csname PY@tok@mh\endcsname{\def\PY@tc##1{\textcolor[rgb]{0.40,0.40,0.40}{##1}}}
\expandafter\def\csname PY@tok@mi\endcsname{\def\PY@tc##1{\textcolor[rgb]{0.40,0.40,0.40}{##1}}}
\expandafter\def\csname PY@tok@il\endcsname{\def\PY@tc##1{\textcolor[rgb]{0.40,0.40,0.40}{##1}}}
\expandafter\def\csname PY@tok@mo\endcsname{\def\PY@tc##1{\textcolor[rgb]{0.40,0.40,0.40}{##1}}}
\expandafter\def\csname PY@tok@ch\endcsname{\let\PY@it=\textit\def\PY@tc##1{\textcolor[rgb]{0.25,0.50,0.50}{##1}}}
\expandafter\def\csname PY@tok@cm\endcsname{\let\PY@it=\textit\def\PY@tc##1{\textcolor[rgb]{0.25,0.50,0.50}{##1}}}
\expandafter\def\csname PY@tok@cpf\endcsname{\let\PY@it=\textit\def\PY@tc##1{\textcolor[rgb]{0.25,0.50,0.50}{##1}}}
\expandafter\def\csname PY@tok@c1\endcsname{\let\PY@it=\textit\def\PY@tc##1{\textcolor[rgb]{0.25,0.50,0.50}{##1}}}
\expandafter\def\csname PY@tok@cs\endcsname{\let\PY@it=\textit\def\PY@tc##1{\textcolor[rgb]{0.25,0.50,0.50}{##1}}}

\def\PYZbs{\char`\\}
\def\PYZus{\char`\_}
\def\PYZob{\char`\{}
\def\PYZcb{\char`\}}
\def\PYZca{\char`\^}
\def\PYZam{\char`\&}
\def\PYZlt{\char`\<}
\def\PYZgt{\char`\>}
\def\PYZsh{\char`\#}
\def\PYZpc{\char`\%}
\def\PYZdl{\char`\$}
\def\PYZhy{\char`\-}
\def\PYZsq{\char`\'}
\def\PYZdq{\char`\"}
\def\PYZti{\char`\~}
% for compatibility with earlier versions
\def\PYZat{@}
\def\PYZlb{[}
\def\PYZrb{]}
\makeatother


    % Exact colors from NB
    \definecolor{incolor}{rgb}{0.0, 0.0, 0.5}
    \definecolor{outcolor}{rgb}{0.545, 0.0, 0.0}



    
    % Prevent overflowing lines due to hard-to-break entities
    \sloppy 
    % Setup hyperref package
    \hypersetup{
      breaklinks=true,  % so long urls are correctly broken across lines
      colorlinks=true,
      urlcolor=urlcolor,
      linkcolor=linkcolor,
      citecolor=citecolor,
      }
    % Slightly bigger margins than the latex defaults
    
    \geometry{verbose,tmargin=1in,bmargin=1in,lmargin=1in,rmargin=1in}
    
    

    \begin{document}
    
    
    \maketitle
    
    

    
    \begin{Verbatim}[commandchars=\\\{\}]
{\color{incolor}In [{\color{incolor}1}]:} \PY{k+kn}{import} \PY{n+nn}{pandas} \PY{k}{as} \PY{n+nn}{pd}
        \PY{k+kn}{import} \PY{n+nn}{numpy} \PY{k}{as} \PY{n+nn}{np}
        \PY{k+kn}{import} \PY{n+nn}{matplotlib}\PY{n+nn}{.}\PY{n+nn}{pyplot} \PY{k}{as} \PY{n+nn}{plt}
        \PY{k+kn}{import} \PY{n+nn}{seaborn} \PY{k}{as} \PY{n+nn}{sns} 
        \PY{k+kn}{from} \PY{n+nn}{sklearn}\PY{n+nn}{.}\PY{n+nn}{preprocessing} \PY{k}{import} \PY{n}{StandardScaler}
        \PY{o}{\PYZpc{}}\PY{k}{matplotlib} inline
        
        \PY{k+kn}{from} \PY{n+nn}{GDELT\PYZus{}utils} \PY{k}{import} \PY{n}{GDELT\PYZus{}columns}\PY{p}{,} \PY{n}{usecols}\PY{p}{,} \PY{n}{dtype\PYZus{}dict}\PY{p}{,} \PYZbs{}
                                \PY{n}{cameo\PYZus{}dict}\PY{p}{,} \PY{n}{map\PYZus{}cameo\PYZus{}to\PYZus{}text}\PY{p}{,} \PYZbs{}
                                \PY{n}{state\PYZus{}dict}\PY{p}{,} \PY{n}{mem\PYZus{}usage}\PY{p}{,} \PY{n}{state\PYZus{}heat\PYZus{}map}
\end{Verbatim}


    \section{Introduction}\label{introduction}

    In this report we're going to take a look at a small sample of the data
from the Global Data on Events, Location and Tone project, also known as
GDELT. GDELT provides data on global events, storing such things as the
type of the event, the actors involved, the location and the tone of
news reporting on the event. It thus provides an abstract and data
driven view of the world's affairs. It has been in operation since 2013,
and has data available from January 1, 1979 to the present.

For our sample of the GDELT data, we're going to take a look at the data
available for the USA's presidential election days in 2008 and 2016:
November 4th, 2008 and November 8th, 2016. We're generally here to
survey the data. To guide our exploration, we seek to determine whether
the average tone of media coverage was significantly different between
the two election days.

    \section{Data Preparation}\label{data-preparation}

    We first load in row limited samples of our two datasets. Our first step
is to quickly verify everything looks normal.

    \begin{Verbatim}[commandchars=\\\{\}]
{\color{incolor}In [{\color{incolor}2}]:} \PY{n}{data\PYZus{}2008} \PY{o}{=} \PY{l+s+s2}{\PYZdq{}}\PY{l+s+s2}{./200811.csv}\PY{l+s+s2}{\PYZdq{}}
        \PY{n}{data\PYZus{}2016} \PY{o}{=} \PY{l+s+s2}{\PYZdq{}}\PY{l+s+s2}{./20161108.export.CSV}\PY{l+s+s2}{\PYZdq{}}
        
        \PY{n}{df\PYZus{}2008\PYZus{}l} \PY{o}{=} \PY{n}{pd}\PY{o}{.}\PY{n}{read\PYZus{}csv}\PY{p}{(}\PY{n}{data\PYZus{}2008}\PY{p}{,} \PY{n}{sep}\PY{o}{=}\PY{l+s+s1}{\PYZsq{}}\PY{l+s+se}{\PYZbs{}t}\PY{l+s+s1}{\PYZsq{}}\PY{p}{,} \PY{n}{nrows}\PY{o}{=}\PY{l+m+mi}{1000}\PY{p}{)}
        \PY{n}{df\PYZus{}2016\PYZus{}l} \PY{o}{=} \PY{n}{pd}\PY{o}{.}\PY{n}{read\PYZus{}csv}\PY{p}{(}\PY{n}{data\PYZus{}2016}\PY{p}{,} \PY{n}{sep}\PY{o}{=}\PY{l+s+s1}{\PYZsq{}}\PY{l+s+se}{\PYZbs{}t}\PY{l+s+s1}{\PYZsq{}}\PY{p}{,} \PY{n}{nrows}\PY{o}{=}\PY{l+m+mi}{1000}\PY{p}{)}
        
        \PY{n}{df\PYZus{}2016\PYZus{}l}\PY{o}{.}\PY{n}{head}\PY{p}{(}\PY{p}{)}
\end{Verbatim}


\begin{Verbatim}[commandchars=\\\{\}]
{\color{outcolor}Out[{\color{outcolor}2}]:}    596760615  20151109  201511  2015  2015.8466 Unnamed: 5    Unnamed: 6  \textbackslash{}
        0  596760616  20151109  201511  2015  2015.8466        NaN           NaN   
        1  596760617  20151109  201511  2015  2015.8466        AUT      AUSTRIAN   
        2  596760618  20151109  201511  2015  2015.8466        AUT      AUSTRIAN   
        3  596760619  20151109  201511  2015  2015.8466     AUTJUD      AUSTRIAN   
        4  596760620  20151109  201511  2015  2015.8466  BWAELIGOV  FESTUS MOGAE   
        
          Unnamed: 7  Unnamed: 8 Unnamed: 9  \textbackslash{}
        0        NaN         NaN        NaN   
        1        AUT         NaN        NaN   
        2        AUT         NaN        NaN   
        3        AUT         NaN        NaN   
        4        BWA         NaN        NaN   
        
                                                        {\ldots}                                                \textbackslash{}
        0                                               {\ldots}                                                 
        1                                               {\ldots}                                                 
        2                                               {\ldots}                                                 
        3                                               {\ldots}                                                 
        4                                               {\ldots}                                                 
        
             967537  3.1 Times Square, New York, United States.1 US.1  USNY.1  \textbackslash{}
        0   1662328    3  Los Angeles, California, United States   US    USCA   
        1        AU    1                                 Austria   AU      AU   
        2        IR    1                                 Austria   AU      AU   
        3        AU    1                                 Austria   AU      AU   
        4  -1397372    4          Gaborone, South East, Botswana   BC    BC09   
        
          40.757.1 -73.986.1  967537.1  20161108  \textbackslash{}
        0  34.0522 -118.2440   1662328  20161108   
        1  47.3333   13.3333        AU  20161108   
        2  47.3333   13.3333        AU  20161108   
        3  47.3333   13.3333        AU  20161108   
        4 -24.6464   25.9119  -1397372  20161108   
        
          http://www.wsaw.com/content/news/Candidates-make-final-stops-on-Election-Day-eve-400321591.html  
        0  http://www.rollingstone.com/culture/news/jinx-{\ldots}                                               
        1  https://www.scmagazine.com/swiss-investigators{\ldots}                                               
        2  https://www.scmagazine.com/swiss-investigators{\ldots}                                               
        3  https://www.scmagazine.com/swiss-investigators{\ldots}                                               
        4  http://www.albuquerqueexpress.com/index.php/si{\ldots}                                               
        
        [5 rows x 58 columns]
\end{Verbatim}
            
    \subsection{Fixing Column Names}\label{fixing-column-names}

    We already have a problem! The CSV's do not provide a header. With the
documentation, we can determine that there are 58 columns for data after
April 2013, and 57 before. With this, we can reload our dataframes
properly.

    \begin{Verbatim}[commandchars=\\\{\}]
{\color{incolor}In [{\color{incolor}3}]:} \PY{n}{df\PYZus{}2008\PYZus{}l} \PY{o}{=} \PY{n}{pd}\PY{o}{.}\PY{n}{read\PYZus{}csv}\PY{p}{(}\PY{n}{data\PYZus{}2008}\PY{p}{,} \PY{n}{sep}\PY{o}{=}\PY{l+s+s1}{\PYZsq{}}\PY{l+s+se}{\PYZbs{}t}\PY{l+s+s1}{\PYZsq{}}\PY{p}{,} \PY{n}{names}\PY{o}{=}\PY{n}{GDELT\PYZus{}columns}\PY{p}{[}\PY{p}{:}\PY{l+m+mi}{57}\PY{p}{]}\PY{p}{,} \PY{n}{nrows}\PY{o}{=}\PY{l+m+mi}{1000}\PY{p}{)}
        \PY{n}{df\PYZus{}2016\PYZus{}l} \PY{o}{=} \PY{n}{pd}\PY{o}{.}\PY{n}{read\PYZus{}csv}\PY{p}{(}\PY{n}{data\PYZus{}2016}\PY{p}{,} \PY{n}{sep}\PY{o}{=}\PY{l+s+s1}{\PYZsq{}}\PY{l+s+se}{\PYZbs{}t}\PY{l+s+s1}{\PYZsq{}}\PY{p}{,} \PY{n}{names}\PY{o}{=}\PY{n}{GDELT\PYZus{}columns}\PY{p}{,} \PY{n}{nrows}\PY{o}{=}\PY{l+m+mi}{1000}\PY{p}{)}
\end{Verbatim}


    \subsection{Improving Memory Consumption and
Performance}\label{improving-memory-consumption-and-performance}

    By default, Pandas makes safe but space inefficient choices for data
types by always defaulting to int64/float64/object. We can improve
memory use and load performance when making a dataframe out of the full
CSV's by omitting any extra data that we are not interested in for our
current analysis and, more importantly, by determining optimized data
types for each column.

For the former case, we note by looking at the
\href{http://data.gdeltproject.org/documentation/GDELT-Data_Format_Codebook.pdf}{Data
Format Codebook} that there are multiple columns which encode dates in
different ways. There are also DATEADDED and SOURCEURL fields,
describing the date the record was added to the data set and the source
of the news respectively, which will not have any bearing on our
analysis. Thus we omit these columns.

Next we look into appropriate data types for our columns. For this
analysis we follow the lead of
\href{https://www.dataquest.io/blog/pandas-big-data/}{Using pandas with
large data}. We first filter our dataframes for particular classes of
types: integers/floats/objects. For the purposes of this report, we only
follow the work on a single dataframe but the analysis has been applied
to both.

    \subsection{Optimizing Number Columns}\label{optimizing-number-columns}

    \begin{Verbatim}[commandchars=\\\{\}]
{\color{incolor}In [{\color{incolor}4}]:} \PY{n}{df\PYZus{}2008\PYZus{}l\PYZus{}int} \PY{o}{=} \PY{n}{df\PYZus{}2008\PYZus{}l}\PY{o}{.}\PY{n}{select\PYZus{}dtypes}\PY{p}{(}\PY{n}{include}\PY{o}{=}\PY{p}{[}\PY{l+s+s1}{\PYZsq{}}\PY{l+s+s1}{int64}\PY{l+s+s1}{\PYZsq{}}\PY{p}{]}\PY{p}{)}
        \PY{n}{df\PYZus{}2008\PYZus{}l\PYZus{}int}\PY{o}{.}\PY{n}{columns}
\end{Verbatim}


\begin{Verbatim}[commandchars=\\\{\}]
{\color{outcolor}Out[{\color{outcolor}4}]:} Index(['GLOBALEVENTID', 'SQLDATE', 'MonthYear', 'Year', 'IsRootEvent',
               'EventCode', 'EventBaseCode', 'EventRootCode', 'QuadClass',
               'NumMentions', 'NumSources', 'NumArticles', 'Actor1Geo\_Lat',
               'Actor1Geo\_Long', 'Actor1Geo\_FeatureID', 'Actor2Geo\_Type',
               'Actor2Geo\_FeatureID', 'DATEADDED'],
              dtype='object')
\end{Verbatim}
            
    We can then attempt to apply "downcasting" to these integer columns.
That is, we can use a method of Pandas, to\_numeric, to find the
smallest size integer type which will support all values of the columns.
The available types are int8, int16, int32 and int64, as well as
unsigned variants (only positive numbers) uint8, uint16, uint32 and
uint64. The numbers at the end of each type indicate the number of bits
consumed in order to represent the data type, and thus smaller numbers
correspond with less memory consumption.

The data's documentation also reaveals that all of these columns will
have positive values. (We also find that some of these fields are not
best represented as integers for the purposes of analysis, more on this
below.) We now apply the downcasting into unsigned integer data types
and show a comparison of memory usage.

    \begin{Verbatim}[commandchars=\\\{\}]
{\color{incolor}In [{\color{incolor}5}]:} \PY{n}{converted\PYZus{}int} \PY{o}{=} \PY{n}{df\PYZus{}2008\PYZus{}l\PYZus{}int}\PY{o}{.}\PY{n}{apply}\PY{p}{(}\PY{n}{pd}\PY{o}{.}\PY{n}{to\PYZus{}numeric}\PY{p}{,}\PY{n}{downcast}\PY{o}{=}\PY{l+s+s1}{\PYZsq{}}\PY{l+s+s1}{unsigned}\PY{l+s+s1}{\PYZsq{}}\PY{p}{)}
        
        \PY{n}{num\PYZus{}before}\PY{p}{,} \PY{n}{str\PYZus{}before} \PY{o}{=} \PY{n}{mem\PYZus{}usage}\PY{p}{(}\PY{n}{df\PYZus{}2008\PYZus{}l\PYZus{}int}\PY{p}{)}
        \PY{n}{num\PYZus{}after}\PY{p}{,} \PY{n}{str\PYZus{}after} \PY{o}{=} \PY{n}{mem\PYZus{}usage}\PY{p}{(}\PY{n}{converted\PYZus{}int}\PY{p}{)}
        
        \PY{n+nb}{print}\PY{p}{(}\PY{l+s+s2}{\PYZdq{}}\PY{l+s+s2}{Before downcast: }\PY{l+s+s2}{\PYZdq{}}\PY{p}{,} \PY{n}{str\PYZus{}before}\PY{p}{)}
        \PY{n+nb}{print}\PY{p}{(}\PY{l+s+s2}{\PYZdq{}}\PY{l+s+s2}{After downcast: }\PY{l+s+s2}{\PYZdq{}}\PY{p}{,} \PY{n}{str\PYZus{}after}\PY{p}{)}
        \PY{n+nb}{print}\PY{p}{(}\PY{l+s+s2}{\PYZdq{}}\PY{l+s+s2}{Percent Decrease}\PY{l+s+s2}{\PYZdq{}}\PY{p}{,} \PY{l+m+mi}{1} \PY{o}{\PYZhy{}} \PY{n}{num\PYZus{}after} \PY{o}{/} \PY{n}{num\PYZus{}before}\PY{p}{)}
\end{Verbatim}


    \begin{Verbatim}[commandchars=\\\{\}]
Before downcast:  0.14 MB
After downcast:  0.04 MB
Percent Decrease 0.7287617990005553

    \end{Verbatim}

    We see a 72\% decrease in memory usage in this example. A similar
process to the one we just applied to int columns can be used for float
columns.

    \subsection{Optimizing Object Columns}\label{optimizing-object-columns}

    Optimizing object columns is more involved. Generally an object column
is made up of string values. In some cases, there is nothing we can do
to minimize memory consumption. This generally applies when there are
many unique string values in a column. However, if there are sufficient
repeats, we can use a different numpy data type known as a category.

This data type leverages a map-like structure to translate repeated
strings into less memory intensive integers, and then covertly replaces
the repeated values with these integers. The original values are then
recoverable in this case by following the mapping. This can result in
huge increases in memory efficiency. For a first pass, we apply a
methodology present in our source for this memory optimization which
replaces an object column with category if less than 50\% of the values
are unique.

    \begin{Verbatim}[commandchars=\\\{\}]
{\color{incolor}In [{\color{incolor}6}]:} \PY{n}{df\PYZus{}2008\PYZus{}l\PYZus{}obj} \PY{o}{=} \PY{n}{df\PYZus{}2008\PYZus{}l}\PY{o}{.}\PY{n}{select\PYZus{}dtypes}\PY{p}{(}\PY{n}{include}\PY{o}{=}\PY{p}{[}\PY{l+s+s1}{\PYZsq{}}\PY{l+s+s1}{object}\PY{l+s+s1}{\PYZsq{}}\PY{p}{]}\PY{p}{)}
        \PY{n}{df\PYZus{}2008\PYZus{}l\PYZus{}obj}\PY{o}{.}\PY{n}{columns}
\end{Verbatim}


\begin{Verbatim}[commandchars=\\\{\}]
{\color{outcolor}Out[{\color{outcolor}6}]:} Index(['Actor2Code', 'Actor2Name', 'Actor2CountryCode', 'Actor2KnownGroupCode',
               'Actor2Religion1Code', 'Actor2Religion2Code', 'Actor2Type1Code',
               'Actor2Type2Code', 'Actor2Type3Code', 'Actor2Geo\_FullName',
               'Actor2Geo\_CountryCode', 'Actor2Geo\_ADM1Code', 'ActionGeo\_FullName',
               'ActionGeo\_CountryCode', 'ActionGeo\_ADM1Code'],
              dtype='object')
\end{Verbatim}
            
    \begin{Verbatim}[commandchars=\\\{\}]
{\color{incolor}In [{\color{incolor}7}]:} \PY{n}{converted\PYZus{}obj} \PY{o}{=} \PY{n}{pd}\PY{o}{.}\PY{n}{DataFrame}\PY{p}{(}\PY{p}{)}
        
        \PY{c+c1}{\PYZsh{} Credit to https://www.dataquest.io/blog/pandas\PYZhy{}big\PYZhy{}data/}
        \PY{k}{for} \PY{n}{col} \PY{o+ow}{in} \PY{n}{df\PYZus{}2008\PYZus{}l\PYZus{}obj}\PY{o}{.}\PY{n}{columns}\PY{p}{:}
            \PY{n}{num\PYZus{}unique\PYZus{}values} \PY{o}{=} \PY{n+nb}{len}\PY{p}{(}\PY{n}{df\PYZus{}2008\PYZus{}l\PYZus{}obj}\PY{p}{[}\PY{n}{col}\PY{p}{]}\PY{o}{.}\PY{n}{unique}\PY{p}{(}\PY{p}{)}\PY{p}{)}
            \PY{n}{num\PYZus{}total\PYZus{}values} \PY{o}{=} \PY{n+nb}{len}\PY{p}{(}\PY{n}{df\PYZus{}2008\PYZus{}l\PYZus{}obj}\PY{p}{[}\PY{n}{col}\PY{p}{]}\PY{p}{)}
            \PY{k}{if} \PY{n}{num\PYZus{}unique\PYZus{}values} \PY{o}{/} \PY{n}{num\PYZus{}total\PYZus{}values} \PY{o}{\PYZlt{}} \PY{l+m+mf}{0.5}\PY{p}{:}
                \PY{n}{converted\PYZus{}obj}\PY{o}{.}\PY{n}{loc}\PY{p}{[}\PY{p}{:}\PY{p}{,}\PY{n}{col}\PY{p}{]} \PY{o}{=} \PY{n}{df\PYZus{}2008\PYZus{}l\PYZus{}obj}\PY{p}{[}\PY{n}{col}\PY{p}{]}\PY{o}{.}\PY{n}{astype}\PY{p}{(}\PY{l+s+s1}{\PYZsq{}}\PY{l+s+s1}{category}\PY{l+s+s1}{\PYZsq{}}\PY{p}{)}
            \PY{k}{else}\PY{p}{:}
                \PY{n}{converted\PYZus{}obj}\PY{o}{.}\PY{n}{loc}\PY{p}{[}\PY{p}{:}\PY{p}{,}\PY{n}{col}\PY{p}{]} \PY{o}{=} \PY{n}{df\PYZus{}2008\PYZus{}l\PYZus{}obj}\PY{p}{[}\PY{n}{col}\PY{p}{]}
                
        \PY{n}{num\PYZus{}before}\PY{p}{,} \PY{n}{str\PYZus{}before} \PY{o}{=} \PY{n}{mem\PYZus{}usage}\PY{p}{(}\PY{n}{df\PYZus{}2008\PYZus{}l\PYZus{}obj}\PY{p}{)}
        \PY{n}{num\PYZus{}after}\PY{p}{,} \PY{n}{str\PYZus{}after} \PY{o}{=} \PY{n}{mem\PYZus{}usage}\PY{p}{(}\PY{n}{converted\PYZus{}obj}\PY{p}{)}
        
        \PY{n+nb}{print}\PY{p}{(}\PY{l+s+s2}{\PYZdq{}}\PY{l+s+s2}{Before downcast: }\PY{l+s+s2}{\PYZdq{}}\PY{p}{,} \PY{n}{str\PYZus{}before}\PY{p}{)}
        \PY{n+nb}{print}\PY{p}{(}\PY{l+s+s2}{\PYZdq{}}\PY{l+s+s2}{After downcast: }\PY{l+s+s2}{\PYZdq{}}\PY{p}{,} \PY{n}{str\PYZus{}after}\PY{p}{)}
        \PY{n+nb}{print}\PY{p}{(}\PY{l+s+s2}{\PYZdq{}}\PY{l+s+s2}{Percent Decrease}\PY{l+s+s2}{\PYZdq{}}\PY{p}{,} \PY{l+m+mi}{1} \PY{o}{\PYZhy{}} \PY{n}{num\PYZus{}after} \PY{o}{/} \PY{n}{num\PYZus{}before}\PY{p}{)}
\end{Verbatim}


    \begin{Verbatim}[commandchars=\\\{\}]
Before downcast:  0.74 MB
After downcast:  0.24 MB
Percent Decrease 0.675466344710465

    \end{Verbatim}

    \subsection{Test Final Optimized
Dataframe}\label{test-final-optimized-dataframe}

    We now collate the information revealed by these naive data type passes
as well as our knowledge of the documentation to arrive at our final
data types for the columns we wish to investigate.

    \begin{Verbatim}[commandchars=\\\{\}]
{\color{incolor}In [{\color{incolor}8}]:} \PY{n}{df\PYZus{}2008\PYZus{}l\PYZus{}opti} \PY{o}{=} \PY{n}{pd}\PY{o}{.}\PY{n}{read\PYZus{}csv}\PY{p}{(}\PY{n}{data\PYZus{}2008}\PY{p}{,} \PY{n}{sep}\PY{o}{=}\PY{l+s+s1}{\PYZsq{}}\PY{l+s+se}{\PYZbs{}t}\PY{l+s+s1}{\PYZsq{}}\PY{p}{,} \PY{n}{names}\PY{o}{=}\PY{n}{GDELT\PYZus{}columns}\PY{p}{[}\PY{p}{:}\PY{l+m+mi}{57}\PY{p}{]}\PY{p}{,} \PYZbs{}
                                \PY{n}{usecols}\PY{o}{=}\PY{n}{usecols}\PY{p}{,} \PY{n}{dtype}\PY{o}{=}\PY{n}{dtype\PYZus{}dict}\PY{p}{,} \PYZbs{}
                                \PY{n}{parse\PYZus{}dates}\PY{o}{=}\PY{p}{[}\PY{l+s+s1}{\PYZsq{}}\PY{l+s+s1}{SQLDATE}\PY{l+s+s1}{\PYZsq{}}\PY{p}{]}\PY{p}{,} \PY{n}{nrows}\PY{o}{=}\PY{l+m+mi}{1000}\PY{p}{)}
        \PY{n}{df\PYZus{}2016\PYZus{}l\PYZus{}opti} \PY{o}{=} \PY{n}{pd}\PY{o}{.}\PY{n}{read\PYZus{}csv}\PY{p}{(}\PY{n}{data\PYZus{}2016}\PY{p}{,} \PY{n}{sep}\PY{o}{=}\PY{l+s+s1}{\PYZsq{}}\PY{l+s+se}{\PYZbs{}t}\PY{l+s+s1}{\PYZsq{}}\PY{p}{,} \PY{n}{names}\PY{o}{=}\PY{n}{GDELT\PYZus{}columns}\PY{p}{,} \PYZbs{}
                                \PY{n}{usecols}\PY{o}{=}\PY{n}{usecols}\PY{p}{,} \PY{n}{dtype}\PY{o}{=}\PY{n}{dtype\PYZus{}dict}\PY{p}{,} \PYZbs{}
                                \PY{n}{parse\PYZus{}dates}\PY{o}{=}\PY{p}{[}\PY{l+s+s1}{\PYZsq{}}\PY{l+s+s1}{SQLDATE}\PY{l+s+s1}{\PYZsq{}}\PY{p}{]}\PY{p}{,} \PY{n}{nrows}\PY{o}{=}\PY{l+m+mi}{1000}\PY{p}{)}
        
        \PY{n}{num\PYZus{}before}\PY{p}{,} \PY{n}{str\PYZus{}before} \PY{o}{=} \PY{n}{mem\PYZus{}usage}\PY{p}{(}\PY{n}{df\PYZus{}2008\PYZus{}l}\PY{p}{)}
        \PY{n}{num\PYZus{}after}\PY{p}{,} \PY{n}{str\PYZus{}after} \PY{o}{=} \PY{n}{mem\PYZus{}usage}\PY{p}{(}\PY{n}{df\PYZus{}2008\PYZus{}l\PYZus{}opti}\PY{p}{)}
        
        \PY{n+nb}{print}\PY{p}{(}\PY{l+s+s2}{\PYZdq{}}\PY{l+s+s2}{Before downcast: }\PY{l+s+s2}{\PYZdq{}}\PY{p}{,} \PY{n}{str\PYZus{}before}\PY{p}{)}
        \PY{n+nb}{print}\PY{p}{(}\PY{l+s+s2}{\PYZdq{}}\PY{l+s+s2}{After downcast: }\PY{l+s+s2}{\PYZdq{}}\PY{p}{,} \PY{n}{str\PYZus{}after}\PY{p}{)}
        \PY{n+nb}{print}\PY{p}{(}\PY{l+s+s2}{\PYZdq{}}\PY{l+s+s2}{Percent Decrease}\PY{l+s+s2}{\PYZdq{}}\PY{p}{,} \PY{l+m+mi}{1} \PY{o}{\PYZhy{}} \PY{n}{num\PYZus{}after} \PY{o}{/} \PY{n}{num\PYZus{}before}\PY{p}{)}
\end{Verbatim}


    \begin{Verbatim}[commandchars=\\\{\}]
Before downcast:  1.06 MB
After downcast:  0.68 MB
Percent Decrease 0.36238106842168205

    \end{Verbatim}

    In the end, with optimized data type choices that also take into
consideration the type of analysis we wish to perform, we have a 36\%
reduction in memory usage for our limited samples. And note we can
expect the memory savings to be even greater for the full data frames
assuming we chose our category columns correctly!

    \section{Data Exploration}\label{data-exploration}

    \begin{Verbatim}[commandchars=\\\{\}]
{\color{incolor}In [{\color{incolor}9}]:} \PY{n}{df\PYZus{}2008} \PY{o}{=} \PY{n}{pd}\PY{o}{.}\PY{n}{read\PYZus{}csv}\PY{p}{(}\PY{n}{data\PYZus{}2008}\PY{p}{,} \PY{n}{sep}\PY{o}{=}\PY{l+s+s1}{\PYZsq{}}\PY{l+s+se}{\PYZbs{}t}\PY{l+s+s1}{\PYZsq{}}\PY{p}{,} \PY{n}{names}\PY{o}{=}\PY{n}{GDELT\PYZus{}columns}\PY{p}{[}\PY{p}{:}\PY{l+m+mi}{57}\PY{p}{]}\PY{p}{,} \PYZbs{}
                                \PY{n}{usecols}\PY{o}{=}\PY{n}{usecols}\PY{p}{,} \PY{n}{dtype}\PY{o}{=}\PY{n}{dtype\PYZus{}dict}\PY{p}{,} \PYZbs{}
                                \PY{n}{parse\PYZus{}dates}\PY{o}{=}\PY{p}{[}\PY{l+s+s1}{\PYZsq{}}\PY{l+s+s1}{SQLDATE}\PY{l+s+s1}{\PYZsq{}}\PY{p}{]}\PY{p}{)}
        \PY{n}{df\PYZus{}2016} \PY{o}{=} \PY{n}{pd}\PY{o}{.}\PY{n}{read\PYZus{}csv}\PY{p}{(}\PY{n}{data\PYZus{}2016}\PY{p}{,} \PY{n}{sep}\PY{o}{=}\PY{l+s+s1}{\PYZsq{}}\PY{l+s+se}{\PYZbs{}t}\PY{l+s+s1}{\PYZsq{}}\PY{p}{,} \PY{n}{names}\PY{o}{=}\PY{n}{GDELT\PYZus{}columns}\PY{p}{,} \PYZbs{}
                                \PY{n}{usecols}\PY{o}{=}\PY{n}{usecols}\PY{p}{,} \PY{n}{dtype}\PY{o}{=}\PY{n}{dtype\PYZus{}dict}\PY{p}{,} \PYZbs{}
                                \PY{n}{parse\PYZus{}dates}\PY{o}{=}\PY{p}{[}\PY{l+s+s1}{\PYZsq{}}\PY{l+s+s1}{SQLDATE}\PY{l+s+s1}{\PYZsq{}}\PY{p}{]}\PY{p}{)}
\end{Verbatim}


    \begin{Verbatim}[commandchars=\\\{\}]
{\color{incolor}In [{\color{incolor}10}]:} \PY{n+nb}{print}\PY{p}{(}\PY{n}{mem\PYZus{}usage}\PY{p}{(}\PY{n}{df\PYZus{}2008}\PY{p}{)}\PY{p}{[}\PY{l+m+mi}{1}\PY{p}{]}\PY{p}{)}
         \PY{n+nb}{print}\PY{p}{(}\PY{n}{mem\PYZus{}usage}\PY{p}{(}\PY{n}{df\PYZus{}2016}\PY{p}{)}\PY{p}{[}\PY{l+m+mi}{1}\PY{p}{]}\PY{p}{)}
\end{Verbatim}


    \begin{Verbatim}[commandchars=\\\{\}]
730.73 MB
142.42 MB

    \end{Verbatim}

    We know that we want to filter rows in these two dataframes to include
only events about the United States as well as for the news on the
correct election date. In the former case, we'll use the column
"ActionGeo\_CountryCode" to determine where the action is reported as
having taken place.

    \subsection{Country Code}\label{country-code}

    \begin{Verbatim}[commandchars=\\\{\}]
{\color{incolor}In [{\color{incolor}11}]:} \PY{n}{df\PYZus{}2008}\PY{p}{[}\PY{l+s+s1}{\PYZsq{}}\PY{l+s+s1}{ActionGeo\PYZus{}CountryCode}\PY{l+s+s1}{\PYZsq{}}\PY{p}{]}\PY{o}{.}\PY{n}{unique}\PY{p}{(}\PY{p}{)}
\end{Verbatim}


\begin{Verbatim}[commandchars=\\\{\}]
{\color{outcolor}Out[{\color{outcolor}11}]:} [AF, US, CA, ID, UK, {\ldots}, FK, NT, AV, NF, VC]
         Length: 241
         Categories (240, object): [AF, US, CA, ID, {\ldots}, NT, AV, NF, VC]
\end{Verbatim}
            
    \begin{Verbatim}[commandchars=\\\{\}]
{\color{incolor}In [{\color{incolor}12}]:} \PY{n}{df\PYZus{}2008\PYZus{}us} \PY{o}{=} \PY{n}{df\PYZus{}2008}\PY{o}{.}\PY{n}{loc}\PY{p}{[}\PY{n}{df\PYZus{}2008}\PY{p}{[}\PY{l+s+s1}{\PYZsq{}}\PY{l+s+s1}{ActionGeo\PYZus{}CountryCode}\PY{l+s+s1}{\PYZsq{}}\PY{p}{]} \PY{o}{==} \PY{l+s+s1}{\PYZsq{}}\PY{l+s+s1}{US}\PY{l+s+s1}{\PYZsq{}}\PY{p}{,}\PY{p}{:}\PY{p}{]}\PY{o}{.}\PY{n}{copy}\PY{p}{(}\PY{p}{)}
         \PY{n}{df\PYZus{}2016\PYZus{}us} \PY{o}{=} \PY{n}{df\PYZus{}2016}\PY{o}{.}\PY{n}{loc}\PY{p}{[}\PY{n}{df\PYZus{}2016}\PY{p}{[}\PY{l+s+s1}{\PYZsq{}}\PY{l+s+s1}{ActionGeo\PYZus{}CountryCode}\PY{l+s+s1}{\PYZsq{}}\PY{p}{]} \PY{o}{==} \PY{l+s+s1}{\PYZsq{}}\PY{l+s+s1}{US}\PY{l+s+s1}{\PYZsq{}}\PY{p}{,}\PY{p}{:}\PY{p}{]}\PY{o}{.}\PY{n}{copy}\PY{p}{(}\PY{p}{)}
\end{Verbatim}


    \subsection{Exploring Event Dates}\label{exploring-event-dates}

    As for the dates, we'll first look at what data is present in our
SQLDATE columns for each dataframe.

    \begin{Verbatim}[commandchars=\\\{\}]
{\color{incolor}In [{\color{incolor}13}]:} \PY{n}{ax} \PY{o}{=} \PY{n}{df\PYZus{}2008\PYZus{}us}\PY{p}{[}\PY{l+s+s1}{\PYZsq{}}\PY{l+s+s1}{SQLDATE}\PY{l+s+s1}{\PYZsq{}}\PY{p}{]}\PY{o}{.}\PY{n}{astype}\PY{p}{(}\PY{n+nb}{str}\PY{p}{)}\PY{o}{.}\PY{n}{value\PYZus{}counts}\PY{p}{(}\PY{p}{)} \PYZbs{}
                              \PY{o}{.}\PY{n}{sort\PYZus{}index}\PY{p}{(}\PY{n}{ascending}\PY{o}{=}\PY{k+kc}{False}\PY{p}{)} \PYZbs{}
                              \PY{o}{.}\PY{n}{plot}\PY{o}{.}\PY{n}{barh}\PY{p}{(}\PY{n}{figsize}\PY{o}{=}\PY{p}{(}\PY{l+m+mi}{10}\PY{p}{,}\PY{l+m+mi}{10}\PY{p}{)}\PY{p}{,} \PY{n}{color}\PY{o}{=}\PY{l+s+s1}{\PYZsq{}}\PY{l+s+s1}{lightblue}\PY{l+s+s1}{\PYZsq{}}\PY{p}{)}\PY{p}{;}
                 
         \PY{n}{ax}\PY{o}{.}\PY{n}{annotate}\PY{p}{(}\PY{l+s+s1}{\PYZsq{}}\PY{l+s+s1}{election day}\PY{l+s+s1}{\PYZsq{}}\PY{p}{,} \PY{n}{xy}\PY{o}{=}\PY{p}{(}\PY{l+m+mi}{13500}\PY{p}{,} \PY{l+m+mi}{26}\PY{p}{)}\PY{p}{,} \PY{n}{xytext}\PY{o}{=}\PY{p}{(}\PY{l+m+mi}{15000}\PY{p}{,} \PY{l+m+mi}{22}\PY{p}{)}\PY{p}{,} \PY{n}{size}\PY{o}{=}\PY{l+m+mi}{15}\PY{p}{,}
                     \PY{n}{arrowprops}\PY{o}{=}\PY{n+nb}{dict}\PY{p}{(}\PY{n}{facecolor}\PY{o}{=}\PY{l+s+s1}{\PYZsq{}}\PY{l+s+s1}{black}\PY{l+s+s1}{\PYZsq{}}\PY{p}{,} \PY{n}{shrink}\PY{o}{=}\PY{l+m+mf}{0.05}\PY{p}{)}\PY{p}{)}
         \PY{c+c1}{\PYZsh{} TODO Annotations to point at 11\PYZhy{}04.}
         \PY{c+c1}{\PYZsh{} https://matplotlib.org/users/annotations.html}
\end{Verbatim}


\begin{Verbatim}[commandchars=\\\{\}]
{\color{outcolor}Out[{\color{outcolor}13}]:} Text(15000,22,'election day')
\end{Verbatim}
            
    \begin{center}
    \adjustimage{max size={0.9\linewidth}{0.9\paperheight}}{output_33_1.png}
    \end{center}
    { \hspace*{\fill} \\}
    
    Thus we see a significant uptick in news on and around election day
which matches what we would expect. We also do not see any unexpected
dates in this 2008 dataframe. Now we look at the 2016 dataframe. The
2016 dataframe was loaded from a CSV labeled for the exact day we are
interested in, November 8th, 2016, so this is really just a sanity check
on the data.

    \begin{Verbatim}[commandchars=\\\{\}]
{\color{incolor}In [{\color{incolor}14}]:} \PY{n}{df\PYZus{}2016\PYZus{}us}\PY{p}{[}\PY{l+s+s1}{\PYZsq{}}\PY{l+s+s1}{SQLDATE}\PY{l+s+s1}{\PYZsq{}}\PY{p}{]}\PY{o}{.}\PY{n}{astype}\PY{p}{(}\PY{n+nb}{str}\PY{p}{)}\PY{o}{.}\PY{n}{value\PYZus{}counts}\PY{p}{(}\PY{p}{)} \PYZbs{}
                              \PY{o}{.}\PY{n}{sort\PYZus{}index}\PY{p}{(}\PY{n}{ascending}\PY{o}{=}\PY{k+kc}{False}\PY{p}{)} \PYZbs{}
                              \PY{o}{.}\PY{n}{plot}\PY{o}{.}\PY{n}{barh}\PY{p}{(}\PY{n}{figsize}\PY{o}{=}\PY{p}{(}\PY{l+m+mi}{10}\PY{p}{,}\PY{l+m+mi}{10}\PY{p}{)}\PY{p}{)}\PY{p}{;}
\end{Verbatim}


    \begin{center}
    \adjustimage{max size={0.9\linewidth}{0.9\paperheight}}{output_35_0.png}
    \end{center}
    { \hspace*{\fill} \\}
    
    And it's a good thing we did it! There are other dates floating around
in the dataset which appear to be anomalous. There's at least one 2006
date where appears to be a typo, as well as a 2015 date thrown in this
dataset. This indicates we should be sure to filter our 2016 dataset for
date as well.

    \begin{Verbatim}[commandchars=\\\{\}]
{\color{incolor}In [{\color{incolor}15}]:} \PY{n}{election\PYZus{}day\PYZus{}2008} \PY{o}{=} \PY{n}{np}\PY{o}{.}\PY{n}{datetime64}\PY{p}{(}\PY{l+s+s1}{\PYZsq{}}\PY{l+s+s1}{2008\PYZhy{}11\PYZhy{}04}\PY{l+s+s1}{\PYZsq{}}\PY{p}{)}
         \PY{n}{df\PYZus{}2008\PYZus{}usel} \PY{o}{=} \PY{n}{df\PYZus{}2008\PYZus{}us}\PY{o}{.}\PY{n}{loc}\PY{p}{[}\PY{n}{df\PYZus{}2008\PYZus{}us}\PY{p}{[}\PY{l+s+s1}{\PYZsq{}}\PY{l+s+s1}{SQLDATE}\PY{l+s+s1}{\PYZsq{}}\PY{p}{]} \PY{o}{==} \PY{n}{election\PYZus{}day\PYZus{}2008}\PY{p}{,}\PY{p}{:}\PY{p}{]}\PY{o}{.}\PY{n}{copy}\PY{p}{(}\PY{p}{)}
         \PY{n}{election\PYZus{}day\PYZus{}2016} \PY{o}{=} \PY{n}{np}\PY{o}{.}\PY{n}{datetime64}\PY{p}{(}\PY{l+s+s1}{\PYZsq{}}\PY{l+s+s1}{2016\PYZhy{}11\PYZhy{}08}\PY{l+s+s1}{\PYZsq{}}\PY{p}{)}
         \PY{n}{df\PYZus{}2016\PYZus{}usel} \PY{o}{=} \PY{n}{df\PYZus{}2016\PYZus{}us}\PY{o}{.}\PY{n}{loc}\PY{p}{[}\PY{n}{df\PYZus{}2016\PYZus{}us}\PY{p}{[}\PY{l+s+s1}{\PYZsq{}}\PY{l+s+s1}{SQLDATE}\PY{l+s+s1}{\PYZsq{}}\PY{p}{]} \PY{o}{==} \PY{n}{election\PYZus{}day\PYZus{}2016}\PY{p}{,}\PY{p}{:}\PY{p}{]}\PY{o}{.}\PY{n}{copy}\PY{p}{(}\PY{p}{)}
\end{Verbatim}


    \subsection{AvgTone}\label{avgtone}

    With our data prepared, we can now start looking at the column of
interest we set out to explore, AvgTone. First, let's get an idea of the
range of values for our dataframes.

    \begin{Verbatim}[commandchars=\\\{\}]
{\color{incolor}In [{\color{incolor}16}]:} \PY{n}{fig}\PY{p}{,} \PY{n}{axs} \PY{o}{=} \PY{n}{plt}\PY{o}{.}\PY{n}{subplots}\PY{p}{(}\PY{n}{nrows}\PY{o}{=}\PY{l+m+mi}{1}\PY{p}{,} \PY{n}{ncols}\PY{o}{=}\PY{l+m+mi}{2}\PY{p}{,} \PY{n}{figsize}\PY{o}{=}\PY{p}{(}\PY{l+m+mi}{15}\PY{p}{,}\PY{l+m+mi}{7}\PY{p}{)}\PY{p}{)}
         \PY{n}{fig}\PY{o}{.}\PY{n}{tight\PYZus{}layout}\PY{p}{(}\PY{n}{pad}\PY{o}{=}\PY{l+m+mi}{5}\PY{p}{)}
         
         \PY{n}{plt}\PY{o}{.}\PY{n}{subplot}\PY{p}{(}\PY{l+m+mi}{1}\PY{p}{,}\PY{l+m+mi}{2}\PY{p}{,}\PY{l+m+mi}{1}\PY{p}{)}
         \PY{n}{df\PYZus{}2008\PYZus{}usel}\PY{p}{[}\PY{l+s+s1}{\PYZsq{}}\PY{l+s+s1}{AvgTone}\PY{l+s+s1}{\PYZsq{}}\PY{p}{]}\PY{o}{.}\PY{n}{plot}\PY{p}{(}\PY{n}{kind}\PY{o}{=}\PY{l+s+s1}{\PYZsq{}}\PY{l+s+s1}{box}\PY{l+s+s1}{\PYZsq{}}\PY{p}{)}
         \PY{n}{plt}\PY{o}{.}\PY{n}{ylim}\PY{p}{(}\PY{o}{\PYZhy{}}\PY{l+m+mi}{25}\PY{p}{,} \PY{l+m+mi}{25}\PY{p}{)}
         \PY{n}{plt}\PY{o}{.}\PY{n}{ylabel}\PY{p}{(}\PY{l+s+s2}{\PYZdq{}}\PY{l+s+s2}{Average Tone}\PY{l+s+s2}{\PYZdq{}}\PY{p}{)}
         \PY{n}{plt}\PY{o}{.}\PY{n}{title}\PY{p}{(}\PY{l+s+s2}{\PYZdq{}}\PY{l+s+s2}{Average Tone on Election Day 2008}\PY{l+s+s2}{\PYZdq{}}\PY{p}{)}
         
         \PY{n}{plt}\PY{o}{.}\PY{n}{subplot}\PY{p}{(}\PY{l+m+mi}{1}\PY{p}{,}\PY{l+m+mi}{2}\PY{p}{,}\PY{l+m+mi}{2}\PY{p}{)}
         \PY{n}{df\PYZus{}2016\PYZus{}usel}\PY{p}{[}\PY{l+s+s1}{\PYZsq{}}\PY{l+s+s1}{AvgTone}\PY{l+s+s1}{\PYZsq{}}\PY{p}{]}\PY{o}{.}\PY{n}{plot}\PY{p}{(}\PY{n}{kind}\PY{o}{=}\PY{l+s+s1}{\PYZsq{}}\PY{l+s+s1}{box}\PY{l+s+s1}{\PYZsq{}}\PY{p}{)}
         \PY{n}{plt}\PY{o}{.}\PY{n}{ylim}\PY{p}{(}\PY{o}{\PYZhy{}}\PY{l+m+mi}{25}\PY{p}{,} \PY{l+m+mi}{25}\PY{p}{)}
         \PY{n}{plt}\PY{o}{.}\PY{n}{ylabel}\PY{p}{(}\PY{l+s+s2}{\PYZdq{}}\PY{l+s+s2}{Average Tone}\PY{l+s+s2}{\PYZdq{}}\PY{p}{)}
         \PY{n}{plt}\PY{o}{.}\PY{n}{title}\PY{p}{(}\PY{l+s+s2}{\PYZdq{}}\PY{l+s+s2}{Average Tone on Election Day 2016}\PY{l+s+s2}{\PYZdq{}}\PY{p}{)}\PY{p}{;}
\end{Verbatim}


    \begin{center}
    \adjustimage{max size={0.9\linewidth}{0.9\paperheight}}{output_40_0.png}
    \end{center}
    { \hspace*{\fill} \\}
    
    We notice right away that the tone of news stories coming out in 2008
was substantially higher and with far fewer variance than what was being
seen in 2016.

    \subsection{AvgTone by State}\label{avgtone-by-state}

    We can break this down further and attempt to understand the tone of
news stories reporting on actions in particular states. To do this,
we'll first have to dig a little deeper into the data and look at the
ActionGeo\_ADM1Code column which codes more specified location
information.

    \begin{Verbatim}[commandchars=\\\{\}]
{\color{incolor}In [{\color{incolor}17}]:} \PY{n}{df\PYZus{}2008\PYZus{}usel}\PY{p}{[}\PY{l+s+s1}{\PYZsq{}}\PY{l+s+s1}{ActionGeo\PYZus{}ADM1Code}\PY{l+s+s1}{\PYZsq{}}\PY{p}{]}\PY{o}{.}\PY{n}{unique}\PY{p}{(}\PY{p}{)}
\end{Verbatim}


\begin{Verbatim}[commandchars=\\\{\}]
{\color{outcolor}Out[{\color{outcolor}17}]:} [USDC, USMD, US, USMO, USOR, {\ldots}, USRI, USDE, USWI, USAR, USWY]
         Length: 53
         Categories (53, object): [USDC, USMD, US, USMO, {\ldots}, USDE, USWI, USAR, USWY]
\end{Verbatim}
            
    We see that the codes for US states are provided in the format of US
followed by the 2 letter state code, provided that the event was able to
be localized down to the state level. state\_dict has already been
defined in our GDELT\_utils.py to map the codes to the state name. We
use it to rename our indices and present the mean AvgTone as grouped by
state.

    \begin{Verbatim}[commandchars=\\\{\}]
{\color{incolor}In [{\color{incolor}18}]:} \PY{n}{avgtone\PYZus{}2008} \PY{o}{=} \PY{n}{df\PYZus{}2008\PYZus{}usel}\PY{o}{.}\PY{n}{groupby}\PY{p}{(}\PY{l+s+s1}{\PYZsq{}}\PY{l+s+s1}{ActionGeo\PYZus{}ADM1Code}\PY{l+s+s1}{\PYZsq{}}\PY{p}{)}\PY{p}{[}\PY{l+s+s1}{\PYZsq{}}\PY{l+s+s1}{AvgTone}\PY{l+s+s1}{\PYZsq{}}\PY{p}{]} \PYZbs{}
                     \PY{o}{.}\PY{n}{mean}\PY{p}{(}\PY{p}{)}\PY{p}{[}\PY{n}{np}\PY{o}{.}\PY{n}{sort}\PY{p}{(}\PY{n}{df\PYZus{}2008\PYZus{}usel}\PY{p}{[}\PY{l+s+s1}{\PYZsq{}}\PY{l+s+s1}{ActionGeo\PYZus{}ADM1Code}\PY{l+s+s1}{\PYZsq{}}\PY{p}{]}\PY{o}{.}\PY{n}{unique}\PY{p}{(}\PY{p}{)}\PY{p}{)}\PY{p}{]} \PYZbs{}
                     \PY{o}{.}\PY{n}{rename}\PY{p}{(}\PY{n}{state\PYZus{}dict}\PY{p}{)}
         \PY{n}{avgtone\PYZus{}2016} \PY{o}{=} \PY{n}{df\PYZus{}2016\PYZus{}usel}\PY{o}{.}\PY{n}{groupby}\PY{p}{(}\PY{l+s+s1}{\PYZsq{}}\PY{l+s+s1}{ActionGeo\PYZus{}ADM1Code}\PY{l+s+s1}{\PYZsq{}}\PY{p}{)}\PY{p}{[}\PY{l+s+s1}{\PYZsq{}}\PY{l+s+s1}{AvgTone}\PY{l+s+s1}{\PYZsq{}}\PY{p}{]} \PYZbs{}
                     \PY{o}{.}\PY{n}{mean}\PY{p}{(}\PY{p}{)}\PY{p}{[}\PY{n}{np}\PY{o}{.}\PY{n}{sort}\PY{p}{(}\PY{n}{df\PYZus{}2016\PYZus{}usel}\PY{p}{[}\PY{l+s+s1}{\PYZsq{}}\PY{l+s+s1}{ActionGeo\PYZus{}ADM1Code}\PY{l+s+s1}{\PYZsq{}}\PY{p}{]}\PY{o}{.}\PY{n}{unique}\PY{p}{(}\PY{p}{)}\PY{p}{)}\PY{p}{]} \PYZbs{}
                     \PY{o}{.}\PY{n}{rename}\PY{p}{(}\PY{n}{state\PYZus{}dict}\PY{p}{)}
\end{Verbatim}


    We first show it on a line graph to give a general impression of the
difference of AvgTone across the board between the two election years.
Then we show a state heatmap to provide a better means of identifying
unique states.

    \begin{Verbatim}[commandchars=\\\{\}]
{\color{incolor}In [{\color{incolor}19}]:} \PY{n}{plt}\PY{o}{.}\PY{n}{figure}\PY{p}{(}\PY{n}{figsize}\PY{o}{=}\PY{p}{(}\PY{l+m+mi}{20}\PY{p}{,}\PY{l+m+mi}{10}\PY{p}{)}\PY{p}{)}
         \PY{n}{plt}\PY{o}{.}\PY{n}{plot}\PY{p}{(}\PY{n}{avgtone\PYZus{}2008}\PY{p}{,} \PY{n}{label}\PY{o}{=}\PY{l+s+s2}{\PYZdq{}}\PY{l+s+s2}{Election Day 2008}\PY{l+s+s2}{\PYZdq{}}\PY{p}{)}
         \PY{n}{plt}\PY{o}{.}\PY{n}{plot}\PY{p}{(}\PY{n}{avgtone\PYZus{}2016}\PY{p}{,} \PY{n}{label}\PY{o}{=}\PY{l+s+s2}{\PYZdq{}}\PY{l+s+s2}{Election Day 2016}\PY{l+s+s2}{\PYZdq{}}\PY{p}{)}
         \PY{n}{plt}\PY{o}{.}\PY{n}{xticks}\PY{p}{(}\PY{n}{rotation}\PY{o}{=}\PY{l+m+mi}{90}\PY{p}{)}\PY{p}{;}
         \PY{n}{plt}\PY{o}{.}\PY{n}{ylim}\PY{p}{(}\PY{o}{\PYZhy{}}\PY{l+m+mf}{4.5}\PY{p}{,} \PY{l+m+mi}{8}\PY{p}{)}
         \PY{n}{plt}\PY{o}{.}\PY{n}{legend}\PY{p}{(}\PY{p}{)}
         \PY{n}{plt}\PY{o}{.}\PY{n}{title}\PY{p}{(}\PY{l+s+s2}{\PYZdq{}}\PY{l+s+s2}{Mean AvgTone by State on Election Day 2008 and 2016}\PY{l+s+s2}{\PYZdq{}}\PY{p}{)}\PY{p}{;}
\end{Verbatim}


    \begin{center}
    \adjustimage{max size={0.9\linewidth}{0.9\paperheight}}{output_48_0.png}
    \end{center}
    { \hspace*{\fill} \\}
    
    \begin{Verbatim}[commandchars=\\\{\}]
{\color{incolor}In [{\color{incolor}20}]:} \PY{n}{state\PYZus{}heat\PYZus{}map}\PY{p}{(}\PY{n}{avgtone\PYZus{}2008}\PY{p}{,} \PY{o}{\PYZhy{}}\PY{l+m+mf}{4.5}\PY{p}{,} \PY{l+m+mi}{8}\PY{p}{,} \PY{l+s+s1}{\PYZsq{}}\PY{l+s+s1}{Average Tone of News on Election Day 2008}\PY{l+s+s1}{\PYZsq{}}\PY{p}{)}\PY{p}{;}
\end{Verbatim}


    \begin{center}
    \adjustimage{max size={0.9\linewidth}{0.9\paperheight}}{output_49_0.png}
    \end{center}
    { \hspace*{\fill} \\}
    
    \begin{Verbatim}[commandchars=\\\{\}]
{\color{incolor}In [{\color{incolor}21}]:} \PY{n}{state\PYZus{}heat\PYZus{}map}\PY{p}{(}\PY{n}{avgtone\PYZus{}2016}\PY{p}{,} \PY{o}{\PYZhy{}}\PY{l+m+mf}{4.5}\PY{p}{,} \PY{l+m+mi}{8}\PY{p}{,} \PY{l+s+s1}{\PYZsq{}}\PY{l+s+s1}{Average Tone of News on Election Day 2016}\PY{l+s+s1}{\PYZsq{}}\PY{p}{)}\PY{p}{;}
\end{Verbatim}


    \begin{center}
    \adjustimage{max size={0.9\linewidth}{0.9\paperheight}}{output_50_0.png}
    \end{center}
    { \hspace*{\fill} \\}
    
    \section{Conclusion}\label{conclusion}

    We set out in this exploration to understand whether we could see a
significant different in the average tone of the documents that discuss
events taking place in the USA. It seems very likely that there is a
statistically significant shift in tone in the 8 years. There are a
myriad of things that might be behind this, and it would be worthwhile
to dig deeper into the data to see what similarities can be found
between the years.

Were the events simply drastically different or had the tone itself
changed due to differences in reporting/media agencies?

Does this have something to do with other known sociological effects
that may have happened in the interim 8 years, perhaps shifts in the
partisan landscape of media?

It's not at all clear what might cause such a drastic shift, especially
since 2008 was in the midst of the last recession, and it should prove
interesting to dig deeper into this and find out what happened.

    \section{References and Data Sources}\label{references-and-data-sources}

    \subsubsection{Data Sources}\label{data-sources}

https://www.gdeltproject.org/data.html

http://data.gdeltproject.org/events/200811.zip

http://data.gdeltproject.org/events/20161108.export.CSV.zip

\subsubsection{Data Documentation}\label{data-documentation}

http://data.gdeltproject.org/documentation/GDELT-Data\_Format\_Codebook.pdf

\subsubsection{Code References}\label{code-references}

https://github.com/matplotlib/basemap/blob/master/examples/fillstates.py

https://stackoverflow.com/questions/39742305/how-to-use-basemap-python-to-plot-us-with-50-states

https://www.dataquest.io/blog/pandas-big-data/

    \section{Appendix: AvgTone Boxplot by
State}\label{appendix-avgtone-boxplot-by-state}

    \begin{Verbatim}[commandchars=\\\{\}]
{\color{incolor}In [{\color{incolor}22}]:} \PY{n}{df\PYZus{}2008\PYZus{}usel}\PY{p}{[}\PY{l+s+s1}{\PYZsq{}}\PY{l+s+s1}{ActionGeo\PYZus{}ADM1Code}\PY{l+s+s1}{\PYZsq{}}\PY{p}{]} \PY{o}{=} \PY{n}{df\PYZus{}2008\PYZus{}usel}\PY{o}{.}\PY{n}{loc}\PY{p}{[}\PY{p}{:}\PY{p}{,}\PY{l+s+s1}{\PYZsq{}}\PY{l+s+s1}{ActionGeo\PYZus{}ADM1Code}\PY{l+s+s1}{\PYZsq{}}\PY{p}{]}\PY{o}{.}\PY{n}{cat}\PY{o}{.}\PY{n}{remove\PYZus{}unused\PYZus{}categories}\PY{p}{(}\PY{p}{)}
         \PY{n}{df\PYZus{}2008\PYZus{}usel}\PY{p}{[}\PY{l+s+s1}{\PYZsq{}}\PY{l+s+s1}{ActionGeo\PYZus{}ADM1Code\PYZus{}Readable}\PY{l+s+s1}{\PYZsq{}}\PY{p}{]} \PY{o}{=} \PY{n}{df\PYZus{}2008\PYZus{}usel}\PY{o}{.}\PY{n}{loc}\PY{p}{[}\PY{p}{:}\PY{p}{,}\PY{l+s+s1}{\PYZsq{}}\PY{l+s+s1}{ActionGeo\PYZus{}ADM1Code}\PY{l+s+s1}{\PYZsq{}}\PY{p}{]}\PY{o}{.}\PY{n}{apply}\PY{p}{(}\PY{k}{lambda} \PY{n}{x}\PY{p}{:}\PY{n}{state\PYZus{}dict}\PY{p}{[}\PY{n}{x}\PY{p}{]}\PY{p}{)}
\end{Verbatim}


    \begin{Verbatim}[commandchars=\\\{\}]
{\color{incolor}In [{\color{incolor}23}]:} \PY{n}{plt}\PY{o}{.}\PY{n}{figure}\PY{p}{(}\PY{n}{figsize}\PY{o}{=}\PY{p}{(}\PY{l+m+mi}{10}\PY{p}{,} \PY{l+m+mi}{20}\PY{p}{)}\PY{p}{)}
         \PY{n}{sns}\PY{o}{.}\PY{n}{boxplot}\PY{p}{(}\PY{n}{data}\PY{o}{=}\PY{n}{df\PYZus{}2008\PYZus{}usel}\PY{p}{,}
                     \PY{n}{x}\PY{o}{=}\PY{l+s+s1}{\PYZsq{}}\PY{l+s+s1}{AvgTone}\PY{l+s+s1}{\PYZsq{}}\PY{p}{,} \PY{n}{y}\PY{o}{=}\PY{l+s+s1}{\PYZsq{}}\PY{l+s+s1}{ActionGeo\PYZus{}ADM1Code\PYZus{}Readable}\PY{l+s+s1}{\PYZsq{}}\PY{p}{)}
         \PY{n}{plt}\PY{o}{.}\PY{n}{axvline}\PY{p}{(}\PY{n}{df\PYZus{}2008\PYZus{}usel}\PY{p}{[}\PY{l+s+s1}{\PYZsq{}}\PY{l+s+s1}{AvgTone}\PY{l+s+s1}{\PYZsq{}}\PY{p}{]}\PY{o}{.}\PY{n}{median}\PY{p}{(}\PY{p}{)}\PY{p}{,} \PY{n}{linewidth}\PY{o}{=}\PY{l+m+mi}{5}\PY{p}{,} \PY{n}{alpha}\PY{o}{=}\PY{l+m+mf}{0.5}\PY{p}{)}
         \PY{n}{plt}\PY{o}{.}\PY{n}{xticks}\PY{p}{(}\PY{n}{rotation}\PY{o}{=}\PY{l+m+mi}{90}\PY{p}{)}\PY{p}{;}
\end{Verbatim}


    \begin{center}
    \adjustimage{max size={0.9\linewidth}{0.9\paperheight}}{output_57_0.png}
    \end{center}
    { \hspace*{\fill} \\}
    
    \section{Appendix: Set CSS Style}\label{appendix-set-css-style}

    \begin{Verbatim}[commandchars=\\\{\}]
{\color{incolor}In [{\color{incolor}24}]:} \PY{c+c1}{\PYZsh{} CSS from https://github.com/CamDavidsonPilon/Probabilistic\PYZhy{}Programming\PYZhy{}and\PYZhy{}Bayesian\PYZhy{}Methods\PYZhy{}for\PYZhy{}Hackers/blob/master/Chapter1\PYZus{}Introduction/Ch1\PYZus{}Introduction\PYZus{}PyMC2.ipynb}
         \PY{k+kn}{from} \PY{n+nn}{IPython}\PY{n+nn}{.}\PY{n+nn}{core}\PY{n+nn}{.}\PY{n+nn}{display} \PY{k}{import} \PY{n}{HTML}
         
         \PY{k}{def} \PY{n+nf}{css\PYZus{}styling}\PY{p}{(}\PY{p}{)}\PY{p}{:}
             \PY{n}{styles} \PY{o}{=} \PY{n+nb}{open}\PY{p}{(}\PY{l+s+s2}{\PYZdq{}}\PY{l+s+s2}{styles/custom.css}\PY{l+s+s2}{\PYZdq{}}\PY{p}{,} \PY{l+s+s2}{\PYZdq{}}\PY{l+s+s2}{r}\PY{l+s+s2}{\PYZdq{}}\PY{p}{)}\PY{o}{.}\PY{n}{read}\PY{p}{(}\PY{p}{)}
             \PY{k}{return} \PY{n}{HTML}\PY{p}{(}\PY{n}{styles}\PY{p}{)}
         \PY{n}{css\PYZus{}styling}\PY{p}{(}\PY{p}{)}
\end{Verbatim}


\begin{Verbatim}[commandchars=\\\{\}]
{\color{outcolor}Out[{\color{outcolor}24}]:} <IPython.core.display.HTML object>
\end{Verbatim}
            

    % Add a bibliography block to the postdoc
    
    
    
    \end{document}
